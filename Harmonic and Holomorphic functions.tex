\documentclass[a4paper]{amsart}

\input{newmacros}

\title{Schwarz Reflection and Harmonicity-Holomorphicity}
\begin{document}
\maketitle

This note summarize the Schwarz reflection lemma and the relation between harmonicity and holomorphicity. 
Originally I wanted to understand the condition required to extend a harmonic function past through a line segment, and I tried to understand this using the Schwarz reflection lemma on holomorphic functions. 
This note is incomplete by itself; currently it is just a collection of some relevant facts. 

First, I copy the sequence of statements given in the Stein's Princeton lecture note, with some modifications.
\begin{lem*}
    Let \(\O\) be a domain in \(\CC\) intersecting with the real axis.
    Let \(L = \O\cap \{\Img z = 0\}\), \(\O_+ = \O\cap \{\Img z > 0\}\), and \(\O_- = \O\cap \{\Img z = 0\}\). 
    If \(f^+\colon \O_+\to \CC\) and \(f^-\colon \O_-\to \CC\) are two holomorphic functions, which can be continuously extended to \(L\), and if their continuous extension agrees on \(L\), then there is a holomorphic `pasting' \(f\) of \(f^+\) and \(f^-\) defined on the entire \(\O\). 
\end{lem*}
The proof of this lemma is direct from the Morera's theorem. 
Note that the conclusion of this lemma is quite significant: We can talk about the complex differentiability of \(f\) on \(L\), where \(f\) was not even defined initially. 
In other words, we can talk about the boundary regularity of \(f\). 
In this context, we can summarize that if a holomorphic function, which is analytic only in \(\O^+\) and continuous up to the boundary, is real on \(L\), then actually it is smooth up to the boundary.

One might think that being real on \(L\) is an unnatural condition. 
Definitely, one can relax this condition; we can prove the following corollary.
\begin{cor*}
    With the same setting, say \(f|_L\) is an analytic function near \(z\in L\).
    Then, \(f\) can be extended to a holomorphic function defined in a small disk containing \(z\). 
\end{cor*}
\begin{proof}
    Say \(f|_L(z) = \sum_{n=0}^{\infty} \frac{a_n}{n!}z^n\) near \(z\).
    Then, the function \(g(z) = \sum_{n=0}^{\infty} \frac{a_n}{n!}z^n\) defined in a disk containing \(z\) is holomorphic.
    Further, \(f - g\) is zero on \(L\) intersect this disk, hence by the lemma above it can be extended in the entire disk. 
    Adding \(g\) again we obtain a holomorphic function on the disk, which completes the proof.
\end{proof}
Note that this is also a necessary condition, as any holomorphic function is analytic in a disc that it is defined.

One interesting consequence occurs when we try to relate real harmonic functions with the holomorphic functions. 
To do this, we need to state the following standard fact. 
\begin{lem*}
    Let \(f\colon \RR^2\to \RR\) be a harmonic function. 
    Then, there is a holomorphic function \(g\colon \CC\to \CC\) with \(\im g(z) = f(z)\), where we identify \(\RR^2\) with \(\CC\). 
\end{lem*}
\begin{proof}
    From the harmonicity, there is a function \(\ftil\) satisfying \(\ftil_x = -f_y\) and \(\ftil_y = f_x\), being an exact differential form. 
    Therefore, we can define \(g(z) = f(z) + i\ftil(z)\). 
    The above relation gives \(\frac{d}{d\zbar}g = 0\).
\end{proof}

Now we shall discuss the boundary regularity of harmonic functions. 
Given a smooth function \(g\colon L\to \RR\), if there is a holomorphic function \(f\colon \O_+\to \CC\) which continuously extends to \(g\) on \(L\), then \(g\) should actually be an analytic function, and \(f\) can be holomorphically extended past \(L\). 
To translate this into harmonic functions, we should consider the real part of \(f\), which should match with \(g\) on \(L\), and plus, its imaginary counterpart should vanish on \(L\). 
This means, \(\ftil_x = 0\) using the above notation, i.e. \(f_y = 0\). 
Therefore, we can argue that, if there is a harmonic function \(f\) which is equal to \(g\) on \(L\), and has zero \(y\) derivative on \(L\), then \(f\) can be harmonically extended past \(L\).
We state this fact into a lemma, and prove this in a PDE dialect, without referring to the reflection principle. 

\begin{lem*}
    Let \(h\colon \O_+\to \RR\) be a harmonic function with \(h|_L = g\) and \(h_y = 0\). 
    Then, \(g\) is actually real analytic, and \(h\) can be defined past \(L\) with being harmonic. 
\end{lem*}
\begin{proof}
    We prove that \(h_y\) can be harmonically extended first: Since \(h_y\) is harmonic in \(\O_+\) and identically \(0\) on \(L\), we can negatively reflect this onto \(\O_-\).
    Note that this is sufficiently regular on \(\O_+\cup\O_-\).
    On \(L\), this is infinitely \(x\)-differentiable, and being a reflection, at least twice differentiable in \(y\) as well, especially the second derivative vanishes. 
    Therefore, \(h_y\) is harmonic in \(\O\). 
    Now, since \(h\) is already real analytic on any horizontal segment in \(\O^+\), using integral we can prove that \(g\) is real analytic. 
    Being an \(y\)-integral of harmonic function from a real analytic function, we obtain that \(h\) is harmonic on \(\O\). 
\end{proof}

I want to check the regularity conditions, based on the things that I do in the last note. 
The regularity theorems given in Evans might help. Let's review them. 




\end{document}